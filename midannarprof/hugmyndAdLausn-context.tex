\subsection*{Dæmi 1}
Finnið þau lið sem unnu leiki sína með a.m.k. þriggja marka mun.

\lausn
\begin{lstlisting}
SELECT homeID AS winnerID
FROM Matches
WHERE homeGoals-awayGoals>=3
	UNION
SELECT awayID AS winnerID
FROM Matches
WHERE awayGoals-homeGoals>=3;
\end{lstlisting}
gefur
\begin{Verbatim}[frame=single]
winnerID
----------
BRA
COL
CRO
ESP
FRA
GER
NED
SUI
\end{Verbatim}

\subsection*{Dæmi 2}
Finnið þau lið sem skoruðu mark að loknum venjulegum leiktíma (sem er 90 mínútur).

\lausn
\begin{lstlisting}
SELECT DISTINCT teamID
FROM Goals
WHERE minute>90;
\end{lstlisting}
gefur
\begin{Verbatim}[frame=single]
teamId
----------
BRA
CHI
COL
SUI
GER
ARG
POR
NED
JPN
GRE
NGA
ALG
BEL
USA
\end{Verbatim}

\subsection*{Dæmi 3}
Finnið nöfnin á þeim löndum með bil í nafninu sínu (t.d. 'Suður Kórea').

\lausn
\begin{lstlisting}
SELECT name
FROM Teams
WHERE name LIKE '% %';
\end{lstlisting}
gefur
\begin{Verbatim}[frame=single]
name
----------------------
Costa Rica
Ivory Coast
Bosnia and Herzegovina
Korea Republic
\end{Verbatim}

\subsection*{Dæmi 4}
Finnið nöfnin á þeim liðum sem kepptu í riðli A (Group A).

\lausn
\begin{lstlisting}
SELECT name
FROM Teams
WHERE groupID=(
	SELECT id
	FROM Groups
	WHERE name='A'
);
\end{lstlisting}
gefur
\begin{Verbatim}[frame=single]
name
----------
Brazil
Croatia
Cameroon
Mexico
\end{Verbatim}

\subsection*{Dæmi 5}
Finnið þá leikmenn og nafn á liði þeirra sem skoruðu mark fyrir lið sem lék í riðli B.

\lausn
\begin{lstlisting}
SELECT DISTINCT name,player
FROM Teams,Goals
WHERE teamID=code AND groupID=(
	SELECT id
	FROM Groups
	WHERE name='B'
);
\end{lstlisting}
gefur
\begin{Verbatim}[frame=single]
name        player
----------  ----------
Australia   Cahill
Australia   Jedinak
Chile       Alexis
Chile       Beausejour
Chile       Ch. ArAngu
Chile       Valdivia
Chile       Vargas
Spain       Alonso
Spain       David Vill
Spain       Mata
Spain       Torres
Netherland  Blind
Netherland  De Vrij
Netherland  Fer
Netherland  Huntelaar
Netherland  Memphis
Netherland  Robben
Netherland  Sneijder
Netherland  V. Persie
Netherland  Wijnaldum
\end{Verbatim}

\subsection*{Dæmi 6}
Finnið þau lið skoruðu jafnmörg mörk sem heimalið í ólíkum leikjum.

\lausn
\begin{lstlisting}
SELECT DISTINCT M1.homeID,M2.homeID
FROM Matches AS M1, Matches AS M2 ON M1.homeGoals=M2.homeGoals
WHERE M1.id<>M2.id AND M1.homeID<>M2.homeID;
\end{lstlisting}
gefur
\begin{Verbatim}[frame=single]
homeId      homeId
----------  ------
BRA         CHI
BRA         COL
BRA         FRA
BRA         BIH
MEX         ESP
MEX         URU
MEX         ENG
MEX         GHA
MEX         RUS
MEX         HON
MEX         ARG
MEX         NGA
MEX         BEL
... (mjög langt, líklega er ég að misskilja þessa spurningu)
\end{Verbatim}

\subsection*{Dæmi 7}
Finnið þá leikmenn sem skoruðu tvö eða fleiri mörk í einum leik og þá leiki þar sem þetta gerðist.

\lausn
\begin{lstlisting}
SELECT player,matchID
FROM Goals
GROUP BY matchID,player
HAVING count(*)>=2;
\end{lstlisting}
gefur
\begin{Verbatim}[frame=single]
player      matchId
----------  ----------
Neymar Jr   1
Robben      3
V. Persie   3
Benzema     10
Muller      13
MandZukiC   18
L. Suarez   23
E. Valenci  26
Neymar Jr   33
Jackson M.  37
Shaqiri     41
Messi       43
Musa        43
James       50
Djabou      54
Ozil        54
Kroos       61
Schurrle    61
\end{Verbatim}

\subsection*{Dæmi 8}
Finnið leikmann, lið og mínútu fyrir hvert mark sem var skorað í leik milli Sviss og Frakklands ('SUI' og 'FRA').

\lausn
\begin{lstlisting}
SELECT player,teamID,minute
FROM Goals
WHERE matchid=(
	SELECT id
	FROM Matches
	WHERE (homeID='SUI' AND awayID='FRA') OR (homeID='FRA' AND awayID='SUI')
);
\end{lstlisting}
gefur
\begin{Verbatim}[frame=single]
player      teamId      minute
----------  ----------  ----------
Dzemaili    SUI         81
Xhaka       SUI         87
Giroud      FRA         17
Matuidi     FRA         18
Valbuena    FRA         40
Benzema     FRA         67
Sissoko     FRA         73
\end{Verbatim}

\subsection*{Dæmi 9}
Finnið markamínútuna, þ.e. þá mínútu þar sem flest mörk voru skoruð.

\lausn
\begin{lstlisting}
SELECT minute
FROM Goals
GROUP BY minute
HAVING count(*)=(
	SELECT max(goalsScored)
	FROM (
		SELECT count(*) AS goalsScored
		FROM Goals
		GROUP BY minute
	)
);
\end{lstlisting}
gefur
\begin{Verbatim}[frame=single]
minute
----------
82
\end{Verbatim}

\subsection*{Dæmi 10}
Finnið markahæsta leikmanninn í þýska landsliðinu ('GER') og hve mörg mörk hann skoraði.

\lausn
\begin{lstlisting}
SELECT player,count(*)
FROM Goals
WHERE teamID='GER'
GROUP BY player
HAVING count(*)=(
	SELECT max(goalsScored)
	FROM (
		SELECT count(*) AS goalsScored
		FROM Goals
		WHERE teamID='GER'
		GROUP BY player
	)
);
\end{lstlisting}
gefur
\begin{Verbatim}[frame=single]
player      count(*)
----------  ----------
Muller      5
\end{Verbatim}

\subsection*{Dæmi 11}
Finnið markahæsta leikmann í hverju liði og hve mörg mörk hann skoraði.

\lausn
\begin{lstlisting}
SELECT player,teamID,max(goalsScored)
FROM (
	SELECT teamID,player,count(*) AS goalsScored
	FROM Goals
	GROUP BY teamID,player
	)
GROUP BY teamID;
\end{lstlisting}
gefur
\begin{Verbatim}[frame=single]
player      teamID      max(goalsScored)
----------  ----------  ----------------
Djabou      ALG         3
Messi       ARG         4
Cahill      AUS         2
De Bruyne   BEL         1
Dzeko       BIH         1
Neymar Jr   BRA         4
Alexis      CHI         2
B. Wilfrie  CIV         2
Matip       CMR         1
James       COL         6
Ruiz B.     CRC         2
MandZukiC   CRO         2
E. Valenci  ECU         3
Rooney      ENG         1
Alonso      ESP         1
Benzema     FRA         3
Muller      GER         5
A. Ayew     GHA         2
Samaras     GRE         1
Costly      HON         1
Reza        IRN         1
Balotelli   ITA         1
Honda       JPN         1
H M Son     KOR         1
A Guardado  MEX         1
V. Persie   NED         4
Musa        NGA         2
Nani        POR         1
Kerzhakov   RUS         1
Shaqiri     SUI         3
L. Suarez   URU         2
Dempsey     USA         2
\end{Verbatim}
ATH: Þessi lausn gefur aðeins einn af markahæstum leikmönnunum, ég sá ekki fyrir mér einfalda leið til að birta alla markahæstu leikmenn liðs þegar þeir voru tveir eða fleiri.

\subsection*{Dæmi 12}
Finnið það lið sem vann flesta leiki.

\lausn
\begin{lstlisting}
CREATE TEMP VIEW winnerTable AS
SELECT homeID AS Winner,count(*) AS Wins,1 AS atHome
FROM Matches
WHERE homeGoals>awayGoals
GROUP BY Winner
	UNION
SELECT awayID AS Winner,count(*) AS Wins,0 AS atHome
FROM Matches
WHERE awayGoals>homeGoals
GROUP BY Winner;

SELECT Winner,sum(Wins) AS totalWins
FROM winnerTable
GROUP BY Winner
HAVING totalWins=(
	SELECT max(totalWins)
	FROM (
		SELECT Winner,sum(Wins) AS totalWins
		FROM winnerTable
		GROUP BY Winner
	)
);
\end{lstlisting}
gefur
\begin{Verbatim}[frame=single]
Winner      totalWins
----------  ----------
GER         6
\end{Verbatim}
Önnur lausn sem notar ekki \texttt{CREATE VIEW} (því við lærðum það aldrei):
\begin{lstlisting}
SELECT Winner,sum(Wins) AS totalWins
FROM (
	SELECT homeID AS Winner,count(*) AS Wins,1 AS atHome
	FROM Matches
	WHERE homeGoals>awayGoals
	GROUP BY Winner
		UNION
	SELECT awayID AS Winner,count(*) AS Wins,0 AS atHome
	FROM Matches
	WHERE awayGoals>homeGoals
	GROUP BY Winner
)
GROUP BY Winner
HAVING totalWins=(
	SELECT max(totalWins)
	FROM (
		SELECT Winner,sum(Wins) AS totalWins
		FROM (
			SELECT homeID AS Winner,count(*) AS Wins,1 AS atHome
			FROM Matches
			WHERE homeGoals>awayGoals
			GROUP BY Winner
				UNION
			SELECT awayID AS Winner,count(*) AS Wins,0 AS atHome
			FROM Matches
			WHERE awayGoals>homeGoals
			GROUP BY Winner
			)
		GROUP BY Winner
	)
);
\end{lstlisting}

\subsection*{Dæmi 13}
Finnið þá leiki þar sem sjálfsmark var skorað.

\lausn
\begin{lstlisting}
SELECT id
FROM Matches, (
	SELECT matchID,teamID,count(*) AS goals
	FROM Goals
	GROUP BY matchID,teamID
) ON Matches.id=matchID AND homeID=teamID
WHERE homeGoals<>goals;
\end{lstlisting}
gefur
\begin{Verbatim}[frame=single]
id
----------
1
10
11
46
53
54
\end{Verbatim}
ATH: Hér geri ég ráð fyrir því að ef sjálfsmark var skorað þá voru mörk liðs ekki jöfn markaskorurum þess liðs í þeim leik. Þetta þýðir samt að t.d. ef bæði lið skora eitt sjálfmark hvort þá myndi það ekki koma fram í þessu svari.